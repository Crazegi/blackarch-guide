%%%%%%%%%%%%%%%%%%%%%%%%%%%%%%%%%%%%%%%%%%%%%%%%%%%%%%%%%%%%%%%%%%%%%%%%%%%%%%%%
%                                                                              %
% BlackArch Linux Przewodnik                                                        %
%                                                                              %
%%%%%%%%%%%%%%%%%%%%%%%%%%%%%%%%%%%%%%%%%%%%%%%%%%%%%%%%%%%%%%%%%%%%%%%%%%%%%%%%

\documentclass[a4paper, oneside, 11pt]{book}

%%% INCLUDES %%%
\renewcommand{\familydefault}{\sfdefault}

\usepackage{array}
\usepackage{color}
\usepackage{enumerate}
\usepackage{fancyhdr}
\usepackage{fancyvrb}
\usepackage{geometry}
\usepackage{graphicx}
\usepackage{html}
\usepackage{hyperref}
\usepackage{ifpdf}
\usepackage{listings}
\usepackage{pstricks}
\usepackage{supertabular}
\usepackage{tocloft}
\usepackage[utf8]{inputenc}

%%% LAYOUT %%%
\setlength{\parindent}{0em}
\setlength{\parskip}{1.5ex plus0.5ex minus0.5ex}
\geometry{left=2.5cm, textwidth=16cm, top=3cm, textheight=25cm, bottom=3cm}
\widowpenalty=2000
\clubpenalty=1000
\frenchspacing
\sloppy
\pagecolor[HTML]{FFFFFF}
\color[HTML]{333333}
\setcounter{tocdepth}{10}
\setcounter{secnumdepth}{10}

\hypersetup{
  pdftitle={BlackArch Linux, The BlackArch Linux Guide},
  pdfsubject={BlackArch Linux, The BlackArch Linux Guide},
  pdfauthor={BlackArch Linux, BlackArch Linux},
  pdfkeywords={BlackArch Linux, Penetration Testing, Security, Hacking, Linux},
  pdfcenterwindow=true,
  colorlinks=true,
  breaklinks=true,
  linkcolor=blue,
  menucolor=blue,
  urlcolor=blue
}

% syntax highlighting
% all options should be set here document wide
\lstset{
backgroundcolor=\color[HTML]{EEEEEE},
frame=single,
basicstyle=\footnotesize\ttfamily,
float,
deletekeywords={return},
otherkeywords={mkdir, curl, sudo, sha1sum, grep, cut, sort, wget, makepkg,
pacman, blackman, chmod},
keywordstyle=\color{orange},
commentstyle=\color{blue},
stringstyle=\color{red},
language=bash,
showspaces=false,
showtabs=false,
tabsize=2
}

%%% HEADER / FOOTER %%%
\setlength{\headheight}{33pt}
\setlength{\headsep}{33pt}
\lhead{{\includegraphics[width=1cm,height=1cm]{images/logo.png}}}
\rhead{The BlackArch Linux Guide}

%%% CUSTOM MACROS %%%
% for html links
\ifpdf\else
\def\href#1#2{\htmladdnormallink{#2}{#1}}
\fi

%------------------%
%  TITLE PAGE      %
%------------------%
\begin{document}
\pagestyle{empty}
\begin{center}
\begin{figure}[htbp]
\centering
\vspace{0.5cm}
\includegraphics[width=8cm]{images/logo.png}
\label{fig:logo}
\end{figure}
\vspace{0.5cm}
\Huge{\textbf{The BlackArch Linux Guide}}\\
\vspace{1cm}
\Large{\color{blue}https://www.blackarch.org/}\\
\vspace{0.5cm}
\end{center}
\newpage
\tableofcontents
\newpage
\pagestyle{fancy}

%------------------%
%  Rozdział 1       %
%------------------%

\chapter{Introduction}

\section{Overview}
Przewodnik BlackArch Linux jest podzielony na kilka następujących części:
\begin{itemize}
\item Wprowadzenie - Zawiera szeroki przegląd, wprowadzenie i dodatkowe przydatne informacje o projekcie
\item Przewodnik Użytkownika - Wszystko, co typowy użytkownik powinien wiedzieć, aby efektywnie korzystać z BlackArcha
\item Przewodnik Dewelopera - Jak pomóc przy tworzeniu BlackArcha
\item Przewodnik Narzędzi - Dogłębne szczegóły narzędzi wraz z przykładowymi zastosowaniami (WIP)
\end{itemize}

\section{Czym jest BlackArch Linux?}
BlackArch to kompletna dystrybucja Linuxa dla testerów penetracyjnych i badaczy cyberbezpieczeństwa.
Wywodzi się z \href{https://www.archlinux.org/}{ArchLinux} i umożliwia użytkowniką instalować komponenty BlackArch indywidualnie lub w grupach bezpośrednio na nim.

Zestaw narzędzi jest dystrybuowany jako \href{https://wiki.archlinux.org/index.php/Unofficial\_User\_Repositories}{nieoficjalne repozytorium użytkowników} Arch Linux, dzięki czemu możesz zainstalować BlackArch na istniejącej instalacji Arch Linux. Pakiety mogą być instalowane indywidualnie lub według kategorii.

Stale rozwijające się repozytorium obecnie zawiera ponad \href{https://www.blackarch.org/tools.html}{2600} narzędzi.
Wszystkie narzędzia są dokładnie testowane przed dodaniem do bazy kodu, aby utrzymać jakość repozytorium.
% should quickly describe the testing methods/code review procedures etc

\section{History of BlackArch Linux}
Wkrótce...

\section{Supported platforms}
Wkrótce...

\section{Get involved}
Możesz skontaktować się z zespołem BlackArch za pośrednictwem tych platform:

Strona Internetowa: \url{https://www.blackarch.org/}

Mail: \href{mailto:team@blackarch.org}{team@blackarch.org}

IRC: \url{irc://irc.freenode.net/blackarch}

Twitter: \url{https://twitter.com/blackarchlinux}

Github: \url{https://github.com/Blackarch/}

Discord: \url{https://discord.com/invite/xMHt8dW}

%------------------%
%  Rozdział 2       %
%------------------%


\chapter{User Guide}

\section{Installation}
Poniższe sekcje pokażą, jak skonfigurować repozytorium BlackArch i instalować pakiety.
BlackArch obsługuje zarówno instalację z repozytorium przy użyciu pakietów binarnych, jak i kompilację oraz instalację ze źródeł.

BlackArch jest kompatybilny z normalnymi instalacjami Arch. Działa jako nieoficjalne repozytorium użytkowników.
Jeśli wolisz obraz ISO, zobacz sekcję \href{https://www.blackarch.org/downloads.html#iso}{ISOs}.

\subsection{Instalacja na ArchLinuxie}
Uruchom \href{https://blackarch.org/strap.sh}{strap.sh} jako root i wykonuj 
dalsze instrukcje. Zobacz poniższy przykład.
\begin{lstlisting}
   curl -O https://blackarch.org/strap.sh
   sha1sum strap.sh # should match: 5ea40d49ecd14c2e024deecf90605426db97ea0c
   sudo chmod +x strap.sh
   sudo ./strap.sh
\end{lstlisting}

Teraz pobierz świeżą kopię głównej listy pakietów i zsynchronizuj pakiety:
\begin{lstlisting}
  sudo pacman -Syyu
\end{lstlisting}


\subsection{Installing packages}
Teraz możesz instalować narzędzia z repozytorium BlackArch.
\begin{enumerate}
\item By wykazać listę wszytkich dostępnych narzędzi, wykonaj:
\begin{lstlisting}
  pacman -Sgg | grep blackarch | cut -d' ' -f2 | sort -u
\end{lstlisting}

\item By zainstalować wsyztkie narzędzia, wykonaj:
\begin{lstlisting}
  pacman -S blackarch
\end{lstlisting}

\item By zainstalować daną kategorię narzędzi, wykonaj
\begin{lstlisting}
  pacman -S blackarch-<category>
\end{lstlisting}

\item By zobaczyć listę kategorii narzędzi BlackArch, wykonaj
\begin{lstlisting}
  pacman -Sg | grep blackarch
\end{lstlisting}

\end{enumerate}

\subsection{Instalowanie pakietów ze źródła}
Jako alternatywna metoda instalacji, możesz zbudować pakiety BlackArch ze źródła. Możesz znaleźć PKGBUILDy na
\href{https://github.com/BlackArch/blackarch/tree/master/packages}{github}. By 
zbudować załe repozytorium, możesz użyć:
\href{https://github.com/BlackArch/blackman}{Blackman} tool.
\begin{itemize}
\item Najpierw musisz zainstalować Blackman.
Jeśli repozytorium pakietów BlackArch jest skonfigurowane na twoim komputerze, możesz zainstalować Blackman:
\begin{lstlisting}
  pacman -S blackman
\end{lstlisting}

\item Możesz skompilować i zainstalować Blackmana ze źródła:
\begin{lstlisting}
  mkdir blackman
  cd blackman
  wget https://raw.github.com/BlackArch/blackarch/master/packages/blackman/PKGBUILD
  # Upewnij się, że PKGBUILD nie został złośliwie zmodyfikowany.
  makepkg -s
\end{lstlisting}

\item Lub możesz zainstalować Blackman z AUR.:
\begin{lstlisting}
  <Niezależnie od używanego narzędzia AUR> -S blackman
\end{lstlisting}

\end{itemize}

\subsection{Podstawowe komendy Blackman} Blackman jest bardzo prosty w użyciu, choć flagi różnią się od tych, które zazwyczaj można spotkać w pacmanie.
Podstawowe użycie zostało opisane poniżej.
\begin{itemize}
\item Pobierz, skompiluj i zainstaluj pakiety:
\begin{lstlisting}
  sudo blackman -i package
\end{lstlisting}

\item Pobierz, skompiluj i zainstaluj całą kategorię:
\begin{lstlisting}
  sudo blackman -g group
\end{lstlisting}

\item Pobierz, skompiluj i zainstaluj wszytkie narzędzia BlackArch:
\begin{lstlisting}
  sudo blackman -a
\end{lstlisting}

\item By wylistować wszytkie kategorie BlackArch:
\begin{lstlisting}
  blackman -l
\end{lstlisting}

\item By wylistować kategorie narzędzi:
\begin{lstlisting}
  blackman -p category
\end{lstlisting}

\end{itemize}

\subsection{Instalowanie z pełnego obrazu ISO, obrazu netinstall lub ArchLinux}
Możesz zainstalować BlackArch Linux z jednego z naszych pełnych obrazów ISO lub obrazów netinstall.
\\Zobacz \url{https://www.blackarch.org/download.html#iso}.
Po uruchomieniu obrazu ISO wymagane są następujące kroki.
\begin{itemize}
\item Zainstaluj pakiet blackarch-installer:
\begin{lstlisting}
  sudo pacman -S blackarch-installer
\end{lstlisting}

\item Wykonaj:
\begin{lstlisting}
  sudo blackarch-install
\end{lstlisting}

\end{itemize}

%------------------%
%  Rozdział 3      %
%------------------%

\chapter{Przewodnik Dewelopera}

\section{System budowy i repozytoria Arch}

Pliki PKGBUILD są skryptami budującymi. Każdy z nich informuje `makepkg(1)`, jak stworzyć pakiet.
Pliki PKGBUILD są napisane w Bashu.

Aby uzyskać więcej informacji, przeczytaj (lub przeskocz) poniższe linki:
\begin{itemize}
\item \href{https://wiki.archlinux.org/index.php/Creating_Packages}{Arch Wiki: Creating Packages}
\item \href{https://wiki.archlinux.org/index.php/Makepkg}{Arch Wiki: makepkg}
\item \href{https://wiki.archlinux.org/index.php/PKGBUILD}{Arch Wiki: PKGBUILD}
\item \href{https://wiki.archlinux.org/index.php/Arch_Packaging_Standards}{Arch Wiki: Arch Packaging Standards}
\end{itemize}

\section{Standardy PKGBUILD BlackArch}
Dla uproszczenia, nasze PKGBUILD są podobne do tych z AUR, z kilkoma drobnymi różnicami opisanymi poniżej.
Każdy pakiet musi przynajmniej należeć do blackarch, dodatkowo może występować wiele nakładających się pakietów należących do różnych grup.

\subsection{Grupy}
Aby umożliwić użytkownikom szybkie i łatwe instalowanie określonego zakresu pakietów, pakiety zostały podzielone na grupy.
Grupy pozwalają użytkownikom na łatwe użycie polecenia "pacman -S <nazwa grupy>", aby pobrać wiele pakietów.

\subsubsection{blackarch}
Grupa blackarch to podstawowa grupa, do której muszą należeć wszystkie pakiety.
Umożliwia to użytkownikom łatwe zainstalowanie każdego pakietu.

Co powinno się tu znajdować: Wszystko.

\subsubsection{blackarch-anti-forensic}
Pakiety używane do przeciwdziałania działalnościom śledczym, w tym szyfrowaniu, steganografii oraz wszelkim operacjom modyfikującym pliki/atrybuty plików.
Obejmuje to narzędzia do pracy z wszelkimi zmianami w systemie w celu ukrywania informacji.

Przykłady: luks, TrueCrypt, Timestomp, dd, ropeadope, secure-delete

\subsubsection{blackarch-anti-forensic}
Pakiety używane do przeciwdziałania działalnościom śledczym, w tym szyfrowaniu, steganografii oraz wszelkim operacjom modyfikującym pliki/atrybuty plików. Obejmuje to narzędzia do pracy z wszelkimi zmianami w systemie w celu ukrywania informacji.

Przykłady: luks, TrueCrypt, Timestomp, dd, ropeadope, secure-delete

\subsubsection{blackarch-automation}
Pakiety używane do automatyzacji narzędzi lub procesów roboczych.

Przykłady: blueranger, tiger, wiffy

\subsubsection{blackarch-backdoor}
Pakiety, które wykorzystują lub otwierają tylne drzwi w już podatnych systemach.

Przykłady: backdoor-factory, rrs, weevely

\subsubsection{blackarch-binary}
Pakiety, które działają na plikach binarnych w pewnej formie.

Przykłady: binwally, packerid

\subsubsection{blackarch-bluetooth}
Pakiety, które wykorzystują standard Bluetooth (802.15.1).

Przykłady: ubertooth, tbear, redfang

\subsubsection{blackarch-code-audit}
Pakiety, które audytują istniejący kod źródłowy w celu analizy podatności.

Przykłady: flawfinder, pscan

\subsubsection{blackarch-cracker}
Pakiety używane do łamania funkcji kryptograficznych, tj. haszy.

Przykłady: hashcat, john, crunch

\subsubsection{blackarch-crypto}
Pakiety, które zajmują się kryptografią, z wyjątkiem łamania.

Przykłady: ciphertest, xortool, sbd

\subsubsection{blackarch-database}
Pakiety związane z eksploatacją baz danych na dowolnym poziomie.

Przykłady: metacoretex, blindsql

\subsubsection{blackarch-debugger}
Pakiety, które pozwalają użytkownikowi na oglądanie, co robi dany program w czasie rzeczywistym.

Przykłady: radare2, shellnoob

\subsubsection{blackarch-decompiler}
Pakiety, które próbują odwrócić skompilowany program do kodu źródłowego.

Przykłady: flasm, jd-gui

\subsubsection{blackarch-defensive}
Pakiety używane do ochrony użytkownika przed złośliwym oprogramowaniem i atakami od innych użytkowników.

Przykłady: arpon, chkrootkit, sniffjoke

\subsubsection{blackarch-disassembler}
Jest to podobne do blackarch-decompiler, i prawdopodobnie wiele programów pasuje do obu kategorii, jednak te pakiety generują wyjście w asemblerze, a nie surowy kod źródłowy.

Przykłady: inguma, radare2

\subsubsection{blackarch-dos}
Pakiety wykorzystujące ataki DoS (Denial of Service).

Przykłady: 42zip, nkiller2

\subsubsection{blackarch-drone}
Pakiety używane do zarządzania fizycznie skonstruowanymi dronami.

Przykłady: meshdeck, skyjack

\subsubsection{blackarch-exploitation}
Pakiety wykorzystujące luki w innych programach lub usługach.

Przykłady: armitage, metasploit, zarp

\subsubsection{blackarch-fingerprint}
Pakiety wykorzystujące sprzęt biometryczny do rozpoznawania linii papilarnych.

Przykłady: dns-map, p0f, httprint

\subsubsection{blackarch-firmware}
Pakiety wykorzystujące luki w oprogramowaniu układowym.

Przykłady: Brak na razie, poprawić jak najszybciej.

\subsubsection{blackarch-forensic}
Pakiety używane do znajdowania danych na fizycznych dyskach lub pamięciach wbudowanych.

Przykłady: aesfix, nfex, wyd

\subsubsection{blackarch-fuzzer}
Pakiety wykorzystujące zasadę testowania fuzzingowego, tj. "rzucanie" losowych wejść na temat, aby zobaczyć, co się stanie.

Przykłady: msf, mdk3, wfuzz

\subsubsection{blackarch-hardware}
Pakiety wykorzystujące lub zarządzające fizycznym sprzętem.

Przykłady: arduino, smali

\subsubsection{blackarch-honeypot}
Pakiety działające jako "honeypoty", tj. programy, które wydają się być podatnymi usługami używanymi do wciągania hakerów w pułapkę.

Przykłady: artillery, bluepot, wifi-honey

\subsubsection{blackarch-keylogger}
Pakiety rejestrujące i przechowujące naciśnięcia klawiszy na innym systemie.

Przykłady: Brak na razie, poprawić jak najszybciej.

\subsubsection{blackarch-malware}
Pakiety uznawane za jakikolwiek rodzaj złośliwego oprogramowania lub wykrywania malware.

Przykłady: malwaredetect, peepdf, yara

\subsubsection{blackarch-misc}
Pakiety, które nie pasują szczególnie do żadnej kategorii.

Przykłady: oh-my-zsh-git, winexe, stompy

\subsubsection{blackarch-mobile}
Pakiety manipulujące platformami mobilnymi.

Przykłady: android-sdk-platform-tools, android-udev-rules

\subsubsection{blackarch-networking}
Pakiety związane z sieciami IP.

Przykłady: arptools, dnsdiag, impacket

\subsubsection{blackarch-nfc}
Pakiety wykorzystujące NFC (komunikację bliskiego zasięgu).

Przykłady: nfcutils

\subsubsection{blackarch-packer}
Pakiety działające na lub związane z pakowaczami.

\textit{Packerzy to programy, które osadzają złośliwe oprogramowanie w innych plikach wykonywalnych.}

Przykłady: packerid

\subsubsection{blackarch-proxy}
Pakiety działające jako proxy, tj. przekierowujące ruch przez inny węzeł w internecie.

Przykłady: burpsuite, ratproxy, sslnuke

\subsubsection{blackarch-recon}
Pakiety aktywnie poszukujące podatnych exploitów w dziczy. Więcej pakietów podobnych do siebie.

Przykłady: canri, dnsrecon, netmask

\subsubsection{blackarch-reversing}
Jest to grupa ogólna dla wszelkich dekompilatorów, disassemblerów lub podobnych programów.

Przykłady: capstone, radare2, zerowine

\subsubsection{blackarch-scanner}
Pakiety skanujące wybrane systemy pod kątem podatności.

Przykłady: scanssh, tiger, zmap

\subsubsection{blackarch-sniffer}
Pakiety związane z analizowaniem ruchu sieciowego.

Przykłady: hexinject, pytactle, xspy

\subsubsection{blackarch-social}
Pakiety, które głównie atakują witryny społecznościowe.

Przykłady: jigsaw, websploit

\subsubsection{blackarch-spoof}
Pakiety próbujące spoofować atakującego, tak aby atakujący nie był widoczny jako atakujący dla ofiary.

Przykłady: arpoison, lans, netcommander

\subsubsection{blackarch-threat-model}
Pakiety używane do raportowania/rekordowania modelu zagrożeń w określonym scenariuszu.

Przykłady: magictree

\subsubsection{blackarch-tunnel}
Pakiety używane do tunelowania ruchu sieciowego w danej sieci.

Przykłady: ctunnel, iodine, ptunnel

\subsubsection{blackarch-unpacker}
Pakiety używane do ekstrakcji wcześniej zapakowanego złośliwego oprogramowania z pliku wykonywalnego.

Przykłady: js-beautify

\subsubsection{blackarch-voip}
Pakiety działające na programach i protokołach VOIP.

Przykłady: iaxflood, rtp-flood, teardown

\subsubsection{blackarch-webapp}
Pakiety działające na aplikacjach dostępnych przez internet.

Przykłady: metoscan, whatweb, zaproxy

\subsubsection{blackarch-windows}
Ta grupa jest przeznaczona dla wszelkich natywnych pakietów Windows, które działają za pośrednictwem wine.

Przykłady: 3proxy-win32, pwdump, winexe

\subsubsection{blackarch-wireless}
Pakiety działające na sieciach bezprzewodowych na każdym poziomie.

Przyk

łady: airpwn, mdk3, wiffy

\section{Struktura repozytorium}
Główne repozytorium BlackArch znajduje się tutaj:
\href{https://github.com/BlackArch/blackarch}{https://github.com/BlackArch/blackarch}.
Istnieje również kilka repozytoriów pomocniczych:
\href{https://github.com/BlackArch}{https://github.com/BlackArch}.

W głównym repozytorium git znajdują się trzy ważne katalogi:

\begin{itemize}
\item docs - Dokumentacja.
\item packages - Pliki PKGBUILD.
\item scripts - Przydatne skrypty.
\end{itemize}

\\subsection{Skrypty}
Oto odniesienie do skryptów w katalogu \verb|scripts/|:

\begin{itemize}
\item baaur - Wkrótce ten skrypt będzie przesyłać pakiety do AUR.
\item babuild - Buduje pakiet.
\item bachroot - Zarządza chrootem do testowania.
\item baclean - Czyści stare pliki .pkg.tar.xz z repozytorium pakietów.
\item baconflict - Wkrótce ten skrypt zastąpi \verb|scripts/conflicts|.
\item bad-files - Znajduje uszkodzone pliki w zbudowanych pakietach.
\item balock - Uzyskuje lub zwalnia blokadę repozytorium pakietów.
\item banotify - Powiadamia IRC o aktualizacjach pakietów.
\item barelease - Publikuje pakiety w repozytorium pakietów.
\item baright - Wyświetla informacje o prawach autorskich BlackArch.
\item basign - Podpisuje pakiety.
\item basign-key - Podpisuje klucz.
\item blackman - Działa trochę jak pacman, ale buduje z git (nie mylić z Blackmanem nrz).
\item check-groups - Sprawdza grupy.
\item checkpkgs - Sprawdza pakiety pod kątem błędów.
\item conflicts - Sprawdza konflikty plików.
\item dbmod - Modyfikuje bazę danych pakietów.
\item depth-list - Tworzy listę posortowaną według głębokości zależności.
\item deptree - Tworzy drzewo zależności, wymieniając tylko pakiety dostarczane przez blackarch.
\item get-blackarch-deps - Pobiera listę zależności blackarch dla pakietu.
\item get-official - Pobiera oficjalne pakiety do wydania.
\item list-loose-packages - Wypisuje pakiety, które nie są w grupach i nie są zależnościami innych pakietów.
\item list-needed - Wypisuje brakujące zależności.
\item list-removed - Wypisuje pakiety, które są w repozytorium pakietów, ale nie w git.
\item list-tools - Wypisuje narzędzia.
\item outdated - Szuka pakietów przestarzałych w repozytorium pakietów w porównaniu do repozytorium git.
\item pkgmod - Modyfikuje zbudowany pakiet.
\item pkgrel - Zwiększa pkgrel w pakiecie.
\item prep - Czyści styl pliku PKGBUILD i znajduje błędy.
\item sitesync - Synchronizuje lokalną kopię repozytorium pakietów z zdalną kopią.
\item size-hunt - Poluje na duże pakiety.
\item source-backup - Tworzy kopię zapasową plików źródłowych pakietów.
\end{itemize}

\section{Wkład w repozytorium}
Ta sekcja pokazuje, jak wnieść wkład do projektu BlackArch Linux. Akceptujemy prośby o dodanie zmian w różnym zakresie, od drobnych poprawek literówek po nowe pakiety.\\W razie pomocy, sugestii lub pytań, prosimy o kontakt z nami.
\\\\
Każdy jest mile widziany do wkładu. Wszystkie wkłady są doceniane.

\subsection{Wymagane samouczki}
Proszę przeczytać poniższe samouczki przed wniesieniem wkładu w projekt:
\begin{itemize}
\item \href{https://wiki.archlinux.org/index.php/Arch\_Packaging\_Standards}{Standardy pakowania w Arch}
\item \href{https://wiki.archlinux.org/index.php/Creating\_Packages}{Tworzenie pakietów}
\item \href{https://wiki.archlinux.org/index.php/PKGBUILD}{PKGBUILD}
\item \href{https://wiki.archlinux.org/index.php/Makepkg}{Makepkg}
\end{itemize}

\subsection{Kroki do wniesienia wkładu}
Aby przesłać swoje zmiany do projektu BlackArchLinux, wykonaj następujące kroki:
\begin{enumerate}
\item Zforkuj repozytorium z
\url{https://github.com/BlackArch/blackarch}
\item Zmodyfikuj odpowiednie pliki (np. PKGBUILD, pliki .patch itp.).
\item Zatwierdź swoje zmiany.
\item Push swoje zmiany.
\item Poproś nas o scalanie zmian, najlepiej za pomocą prośby o pull request.
\end{enumerate}

\subsection{Przykład}
Poniższy przykład demonstruje dodanie nowego pakietu do projektu BlackArch.
Używamy \href{https://github.com/Jguer/yay}{yay} (można również użyć pacaur) do pobrania istniejącego pliku PKGBUILD dla \textbf{nfsshell} z \href{https://aur.archlinux.org/}{AUR} i dostosowania go do naszych potrzeb.

\subsubsection{Pobieranie PKGBUILD}
Pobierz plik \textit{PKGBUILD} za pomocą yay lub pacaur:
\begin{lstlisting}
  user@blackarchlinux $ yay -G nfsshell
  ==> Pobieranie źródeł nfsshell
  x LICENSE
  x PKGBUILD
  x gcc.patch
  user@blackarchlinux $ cd nfsshell/
\end{lstlisting}

\subsubsection{Czyszczenie PKGBUILD}
Oczyść plik \textit{PKGBUILD} i zaoszczędź czas:
\begin{lstlisting}
  user@blackarchlinux nfsshell $ ./blackarch/scripts/prep PKGBUILD
  czyszczenie 'PKGBUILD'...
  rozwijanie tabulatorów...
  usuwanie modeline vim...
  usuwanie komentarza id...
  usuwanie komentarzy o wkładzie i utrzymaniu...
  usuwanie zbędnych pustych linii...
  usuwanie '|| return'...
  usuwanie wiodącej pustej linii...
  usuwanie $pkgname...
  usuwanie białych znaków na końcu...
\end{lstlisting}

\subsubsection{Dostosowanie PKGBUILD}
Dostosuj plik \textit{PKGBUILD}:
\begin{lstlisting}
  user@blackarchlinux nfsshell $ vi PKGBUILD
\end{lstlisting}

\subsubsection{Budowanie pakietu}
Zbuduj pakiet:
\begin{lstlisting}
user@blackarchlinux nfsshell $ makepkg -sf
==> Making package: nfsshell 19980519-1 (Mon Dec  2 17:23:51 CET 2013)
==> Checking runtime dependencies...
==> Checking buildtime dependencies...
==> Retrieving sources...
-> Downloading nfsshell.tar.gz...
% Total    % Received % Xferd  Average Speed   Time    Time     Time
CurrentDload  Upload   Total   Spent    Left  Speed100 29213  100 29213    0
0  48150      0 --:--:-- --:--:-- --:--:-- 48206
-> Found gcc.patch
-> Found LICENSE
...
<mnóstwo procesu budowania i wyników kompilatora>
...
==> Leaving fakeroot environment.
==> Finished making: nfsshell 19980519-1 (Mon Dec  2 17:23:53 CET 2013)
\end{lstlisting}

\subsubsection{Instalacja i testowanie pakietu}
Zainstaluj i przetestuj pakiet:
\begin{lstlisting}
  user@blackarchlinux nfsshell $ pacman -U nfsshell-19980519-1-x86_64.pkg.tar.xz
  user@blackarchlinux nfsshell $ nfsshell # przetestuj
\end{lstlisting}

\subsubsection{Dodanie, zatwierdzenie i wysłanie pakietu}
Dodaj, zatwierdź i wyślij pakiet:
\begin{lstlisting}
user@blackarchlinux nfsshell $ cd /blackarchlinux/packages
user@blackarchlinux ~/blackarchlinux/packages $ mv ~/nfsshell .
user@blackarchlinux ~/blackarchlinux/packages $ git commit -am nfsshell && git push
\end{lstlisting}

\subsubsection{Utworzenie pull request}
Utwórz pull request na \href{https://github.com/}{github.com}
\begin{lstlisting}
  firefox https://github.com/<contributor>/blackarchlinux
\end{lstlisting}

\subsubsection{Dodanie zdalnego repozytorium dla upstream}
Rozsądnym krokiem, jeśli pracujesz nad upstream i forkiem, jest pobranie własnego forka i dodanie głównego repozytorium ba jako zdalnego repozytorium:
\begin{lstlisting}
  user@blackarchlinux ~/blackarchlinux $ git remote -v
  origin <the url of your fork> (fetch)
  origin <the url of your fork> (push)
  user@blackarchlinux ~/blackarchlinux $ git remote add upstream https://github.com/blackarch/blackarch
  user@blackarchlinux ~/blackarchlinux $ git remote -v
  origin <the url of your fork> (fetch)
  origin <the url of your fork> (push)
  upstream https://github.com/blackarch/blackarch (fetch)
  upstream https://github.com/blackarch/blackarch (push)
\end{lstlisting}

Domyślnie, git powinien pushować bezpośrednio do origin, ale upewnij się, że twój konfiguracja git jest poprawnie ustawiona.
To nie będzie problemem, chyba że masz prawa do zatwierdzania, ponieważ bez nich nie będziesz mógł pushować do upstreama.

Jeśli masz uprawnienia do commitowania, możesz mieć więcej szczęścia używając \textit{git@github.com:blackarch/blackarch.git}, ale to zależy od ciebie.

\subsection{Prośby}
\begin{enumerate}
\item Nie dodawaj komentarzy \textbf{Maintainer} lub \textbf{Contributor} do plików \textit{PKGBUILD}. Dodaj nazwiska opiekunów i współautorów do sekcji AUTHORS w przewodniku BlackArch.
\item Dla zachowania spójności, prosimy o stosowanie ogólnego stylu innych plików \textit{PKGBUILD} w repozytorium i używanie wcięcia dwoma spacjami.
\end{enumerate}

\subsection{Ogólne wskazówki}
\href{http://wiki.archlinux.org/index.php/Namcap}{namcap} może sprawdzać pakiety pod kątem błędów.

%------------------%
%  Rozdział 4      %
%------------------%

\chapter{Przewodnik Narzędzi}
Wkrótce...

\section{Wkrótce...}
Wkrótce...

%%% APPENDIX %%%
\appendix
\include{latex/appendix-en}

\end{document}

%%% EOF %%%
